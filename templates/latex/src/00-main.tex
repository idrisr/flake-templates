\documentclass[openany, 12pt]{book}
\input{preamble}
\title{My Great Document}
\author{Idris}
\date{September 2024}

%chktex-file 1
% chktex-file 2999999

% \newglossaryentry{mesa} {
% name=mesa,
% description={
% Collection of userspace GPU drivers that implement APIs like OpenGL,
% Vulkan, and OpenCL, translating them into GPU commands on Linux
% }
% }

\makeglossaries

\begin{document}
\maketitle{}
\tableofcontents

\listoffigures
\listoftables

\chapter{Type Inference}

\begin{enumerate}[label = {(\alph*)}]
  \item asdf
  \item asdf
\end{enumerate}

\begin{table}[h]
  \centering
  \rowcolors{2}{blue!5}{white}
  \begin{tabular}{clll}
    \toprule
    \textbf{\#} & \textbf{Subsystem}    & \textbf{Dir}                         & \textbf{Why It Matters}                                  \\
    \midrule
    1           & Process               & \texttt{kernel/}                     & Scheduling, syscalls, signals, process creation          \\
    2           & Memory Management     & \texttt{mm/}                         & Paging, virtual memory, \texttt{mmap}, slab/slub         \\
    3           & VFS (Virtual FS)      & \texttt{fs/}                         & interface for all filesystems                            \\
    4           & Device Drivers        & \texttt{drivers/}                    & all I/O: input, storage, audio, video                    \\
    5           & Networking Stack      & \texttt{net/}                        & Interfaces, routing, TCP/IP stack, socket APIs           \\
    6           & Block I/O             & \texttt{block/}                      & Disks, I/O schedulers, \texttt{bio} structs, filesystems \\
    7           & Interrupts \& IRQs    & \texttt{kernel/irq/}                 & Device comm, IRQ handling, latency control               \\
    8           & ALSA (Audio)          & \texttt{sound/}                      & Core kernel audio, used by PipeWire                      \\
    9           & cgroups \& Namespaces & \texttt{kernel/}, \texttt{fs/}       & Containers, isolation, virtualization                    \\
    10          & Init \& Boot          & \texttt{init/}, \texttt{init/main.c} & Kernel startup, early init, process 1 launch             \\
    \bottomrule
  \end{tabular}
  \caption{Top 10 Linux Kernel Subsystems to Learn First, with Source Tree Locations}
\end{table}


\setboxcounter{intuition}{9}
\begin{intuition}{Type Inference}
  The essense of type inference is walking the abstract syntax tree and
  collect constraints we must satisfy. Therefore we would reformulate the
  problem to so-called unification problem.
\end{intuition}

\begin{definition}{Type Inference}{}
  Figuring out whether an expression $e$ is well-typed, and if so, what it's
  type $T$ is $e: T$.
\end{definition}

\begin{definition}{Type Inference By Hand}{}
  We can approximate this algorithm by
  \begin{enumerate}
    \item assigning a fresh type variable to each bound name
    \item assigning a fresh type variable to the function return value
    \item solving type constraints until there is nothing more to do.
  \end{enumerate}
\end{definition}

\section{Example}
\begin{haskell}{}
myFun w x y z = if y then w : tail x else y : z
\end{haskell}

\section{Example}

\chapter{Functor\index{Functor}}
\gls{KMS}

\clearpage
\printglossaries

% \printbibliography{}
\printindex{}
\end{document}
