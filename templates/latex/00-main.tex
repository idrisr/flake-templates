\documentclass[12pt]{article}
\usepackage{hippoidP}
\usepackage{standalone}
\usepackage{dirtytalk}
\usepackage[T1]{fontenc}
\usepackage{tikz}
\usepackage{minted}

\usepackage{makeidx}
\makeindex
\usepackage[hyperindex]{hyperref}

\usetikzlibrary{arrows}
\usetikzlibrary{arrows.meta}
\usetikzlibrary{automata}
\usetikzlibrary{calc}
\usetikzlibrary{fit}
\usetikzlibrary{petri}
\usetikzlibrary{positioning}
\usetikzlibrary{shapes.geometric}

\usepackage[printsolution=true]{exercises}
\usepackage[margin=1.2in]{geometry}% Sets 1in margins.
\usepackage{float}
\setlength{\parindent}{0pt}

\docsetup{Category Theory / Haskell}{Programming with Categories}{April 8, 2024}

\theoremstyle{definition}
\newtheorem{definition}{Definition}[section]

\begin{document}
\tableofcontents
\section{Categories, Types, and Functions}
\subsection{Two fundamental ideas: sets and functions}

\begin{minted}{c}
int main() {
    printf("hello, world");
    return 0;
}
\end{minted}

here is some inline code \mint{python}|import this|.

% \inputminted{haskell}{../02-composition/src/Comp.hs}

\printindex
\end{document}
